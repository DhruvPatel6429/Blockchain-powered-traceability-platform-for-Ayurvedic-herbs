% IEEE Conference Paper Template for Overleaf
\documentclass[conference]{IEEEtran}

% Packages
\usepackage{cite}
\usepackage{amsmath,amssymb,amsfonts}
\usepackage{algorithmic}
\usepackage{graphicx}
\usepackage{textcomp}
\usepackage{xcolor}
\usepackage{url}
\usepackage{hyperref}

% Correct bad hyphenation
\hyphenation{op-tical net-works semi-conduc-tor}

\begin{document}

% Paper Title
\title{Blockchain-Based Geo-Tagged Traceability System for Sustainable Ayurvedic Herbal Supply Chains}

% Author Information
\author{
\IEEEauthorblockN{Dhruv Patel\IEEEauthorrefmark{1}, 
Kanav Bansal\IEEEauthorrefmark{2}, 
Aayush Deore\IEEEauthorrefmark{3}, 
Naren A Kittur\IEEEauthorrefmark{4}}
\IEEEauthorblockA{\IEEEauthorrefmark{1}Department of Computer Science, 
R.V. College of Engineering, Bangalore, India\\
Email: dhruvpatel.cs24@rvce.edu.in}
\IEEEauthorblockA{\IEEEauthorrefmark{2}Department of Computer Science,
R.V. College of Engineering, Bangalore, India\\
Email: kanavbansal.cs24@rvce.edu.in}
\IEEEauthorblockA{\IEEEauthorrefmark{3}Department of Computer Science,
R.V. College of Engineering, Bangalore, India\\
Email: aayushdeore.cs24@rvce.edu.in}
\IEEEauthorblockA{\IEEEauthorrefmark{4}Department of Computer Science,
R.V. College of Engineering, Bangalore, India\\
Email: narenakittur.cs24@rvce.edu.in}
}

% Make the title
\maketitle

% Abstract
\begin{abstract}
The Ayurvedic herbal supply chain in India faces significant challenges including poor traceability, adulteration, lack of geographic provenance documentation, over-harvesting, and diminished consumer trust. Existing centralized database systems fail to provide immutability, transparency, and comprehensive auditability across fragmented stakeholders such as small-scale farmers, wild collectors, testing laboratories, processors, manufacturers, regulators, and end consumers. This paper presents a permissioned blockchain-based traceability system that integrates geo-tagging and smart contract validation to record every stage of an Ayurvedic herb's lifecycle from collection to final packaged product. The proposed system employs SHA-256 cryptographic hash chaining for immutable event recording, GPS-based location tagging for provenance verification, and QR-code-enabled consumer transparency. A proof-of-concept implementation was developed using FastAPI backend, React frontend, and MongoDB for off-chain storage, demonstrating feasibility for small-scale deployment. The system ensures authenticity, sustainability compliance, and regulatory auditability while maintaining scalability for multiple herb species and stakeholder participation. Initial pilot results indicate successful blockchain integrity verification, sub-second event recording latency, and effective supply chain transparency for Ayurvedic herb batches.
\end{abstract}

% Keywords
\begin{IEEEkeywords}
Blockchain, Supply Chain Traceability, Ayurveda, Geo-Tagging, Smart Contracts, Healthcare Technology, Sustainability
\end{IEEEkeywords}

% Introduction
\section{Introduction}

\subsection{Background}
The Indian Ayurvedic medicine industry, valued at over USD 18 billion, relies heavily on the sourcing of authentic medicinal herbs from diverse geographic locations across the country. Traditional supply chains for Ayurvedic herbs involve multiple stakeholders including rural collectors, middlemen, testing laboratories, processing units, manufacturers, and distributors. This fragmented ecosystem suffers from critical issues of transparency, authenticity verification, and sustainable harvesting practices. The lack of verifiable provenance information has led to widespread adulteration, with studies indicating that up to 20 percent of herbal products in the market contain substitute or contaminated materials.

\subsection{Problem Statement}
The Ayurvedic herbal supply chain faces several fundamental challenges:

\begin{enumerate}
\item \textbf{Traceability Gap:} Inability to track herbs from their geographic origin to the final consumer product
\item \textbf{Authenticity Concerns:} Widespread adulteration and species substitution without detection mechanisms
\item \textbf{Regulatory Compliance:} Difficulty in meeting export certification requirements and quality standards
\item \textbf{Sustainability Crisis:} Over-harvesting of endangered species due to lack of demand monitoring
\item \textbf{Consumer Trust Deficit:} Limited access to verifiable information about product origins and quality testing
\item \textbf{Documentation Fragmentation:} Paper-based records across disconnected stakeholders leading to data loss and manipulation
\end{enumerate}

Centralized database systems, while providing some digitization, remain vulnerable to single-point failures, unauthorized modifications, and lack of transparency across competing stakeholders in the supply chain.

\subsection{Research Objectives}
This work aims to address the aforementioned challenges through the following objectives:

\begin{enumerate}
\item Design and implement a blockchain-based traceability system for Ayurvedic herbs that ensures immutability and transparency
\item Integrate GPS geo-tagging capabilities to establish verifiable geographic provenance
\item Develop smart contract logic for automated validation and compliance checking
\item Enable consumer access to complete supply chain information through QR code scanning
\item Evaluate the system's feasibility for deployment in resource-constrained environments
\item Demonstrate proof-of-concept implementation for pilot herb species
\end{enumerate}

\subsection{Contributions}
The primary contributions of this research are:

\begin{enumerate}
\item A novel blockchain architecture specifically designed for Ayurvedic herbal supply chain traceability
\item Integration of geo-location tagging with cryptographic hash chaining for tamper-evident provenance records
\item Event-based data model supporting collection, processing, testing, and distribution tracking
\item Practical implementation demonstrating feasibility with existing technology stack
\item Consumer-facing transparency interface through QR code-based provenance access
\item Performance evaluation and identification of deployment challenges for small-scale stakeholders
\end{enumerate}

% Related Work
\section{Related Work}

\subsection{Blockchain in Supply Chain Management}
Blockchain technology has gained significant traction in supply chain applications due to its inherent properties of immutability, transparency, and distributed consensus. Research has demonstrated successful implementations in food safety, pharmaceutical tracking, and agricultural product certification \cite{nakamoto2008bitcoin, kshetri2018blockchain}. Studies have shown that permissioned blockchains, such as Hyperledger Fabric, offer better performance and privacy controls compared to public blockchains for enterprise supply chain applications \cite{hyperledger2020}. However, most existing implementations focus on large-scale industrial supply chains with well-established infrastructure.

\subsection{Agricultural and Food Traceability Systems}
Previous work in agricultural traceability has primarily focused on food safety and organic certification. Systems have been developed for tracking coffee beans, olive oil, wine, and organic vegetables using RFID tags, blockchain, and IoT sensors \cite{tian2016agri}. These systems have demonstrated the technical feasibility of farm-to-fork traceability but often require significant infrastructure investment and stakeholder coordination. The unique challenges of wild-collected medicinal herbs, including diverse collection points and seasonal availability, remain inadequately addressed in existing literature.

\subsection{Pharmaceutical Supply Chain Security}
The pharmaceutical industry has explored blockchain solutions for combating counterfeit drugs and ensuring cold chain integrity \cite{azaria2016medrec}. Systems have been proposed using smart contracts for automated expiry tracking and temperature monitoring. However, these solutions are primarily designed for manufactured medicines with controlled production environments, rather than naturally sourced botanical materials with variable quality parameters.

\subsection{Herbal Medicine Authentication}
Research in herbal medicine authentication has primarily focused on analytical techniques such as DNA barcoding, chromatography, and spectroscopy for species identification. While these methods provide scientific validation, they lack integration with supply chain tracking systems. Few studies have attempted to combine analytical authentication with blockchain-based provenance tracking.

\subsection{Research Gaps}
The literature review reveals several gaps:

\begin{enumerate}
\item Limited research on blockchain applications specifically for traditional medicine supply chains
\item Absence of integrated geo-location and quality testing frameworks for herbal products
\item Lack of low-infrastructure solutions suitable for rural herb collectors
\item Insufficient focus on consumer accessibility to provenance information
\item Limited evaluation of blockchain systems for small-batch, high-diversity botanical products
\end{enumerate}

This work addresses these gaps by presenting a targeted solution for Ayurvedic herbal supply chain traceability.

% System Architecture
\section{System Architecture}

\subsection{Architecture Overview}
The proposed system employs a three-layer architecture designed for modularity, scalability, and ease of deployment:

\begin{enumerate}
\item \textbf{Data Capture Layer:} Mobile-first interface for stakeholders to record supply chain events
\item \textbf{Blockchain Layer:} Cryptographic hash-chaining mechanism for immutable event recording
\item \textbf{Storage Layer:} Hybrid approach combining on-chain metadata with off-chain detailed records
\item \textbf{Application Layer:} Consumer-facing and stakeholder interfaces for data access and visualization
\end{enumerate}

\subsection{Permissioned Blockchain Design Rationale}
The system employs a permissioned blockchain approach rather than a public blockchain for several critical reasons:

\begin{enumerate}
\item \textbf{Performance Requirements:} Sub-second transaction confirmation needed for practical field operations
\item \textbf{Privacy Considerations:} Sensitive business information and collector identities require access control
\item \textbf{Cost Efficiency:} Elimination of cryptocurrency transaction fees unsuitable for low-margin agricultural products
\item \textbf{Regulatory Compliance:} Identifiable participants necessary for legal accountability and export certification
\item \textbf{Stakeholder Coordination:} Known network participants facilitate trust-building and dispute resolution
\end{enumerate}

\subsection{System Components}

\subsubsection{Data Capture Layer}
The data capture layer consists of mobile applications and web interfaces enabling stakeholders to record events. Key features include:
\begin{itemize}
\item GPS-based automatic location tagging using device sensors
\item Offline data capture with subsequent synchronization capabilities
\item Photo and document attachment support for quality certificates
\item User authentication and role-based access control
\item Standardized data entry forms with validation rules
\end{itemize}

\subsubsection{Blockchain Layer}
The blockchain layer implements cryptographic hash chaining to ensure data immutability:
\begin{itemize}
\item Each event is assigned a unique identifier and timestamp
\item SHA-256 hashing algorithm creates cryptographic digest of event data
\item Previous event hash is included in current event, forming an immutable chain
\item Block number sequencing maintains temporal ordering
\item Merkle tree structures enable efficient integrity verification
\end{itemize}

\subsubsection{Off-Chain Storage Layer}
Large files and detailed records are stored off-chain to optimize blockchain performance:
\begin{itemize}
\item MongoDB database for structured event data and batch information
\item Document storage for quality certificates and test reports
\item Image storage for geo-tagged photographs of collection sites
\item Content addressing using IPFS-compatible hash references for integrity verification
\end{itemize}

\subsubsection{Application Layer}
The application layer provides interfaces for different stakeholder roles:
\begin{itemize}
\item Collection interface for farmers and wild collectors
\item Processing interface for drying, cleaning, and grinding facilities
\item Testing interface for quality assurance laboratories
\item Dashboard for manufacturers and supply chain managers
\item Consumer scanning interface for QR code-based provenance access
\end{itemize}

\subsection{Technology Stack}
The proof-of-concept implementation utilizes:
\begin{itemize}
\item \textbf{Backend:} FastAPI (Python) for RESTful API services
\item \textbf{Frontend:} React with Tailwind CSS for responsive user interfaces
\item \textbf{Database:} MongoDB for flexible document storage
\item \textbf{Blockchain Simulation:} SHA-256 hash chaining with Python cryptography libraries
\item \textbf{Deployment:} Containerized architecture for scalability
\end{itemize}

% Methodology
\section{Methodology}

\subsection{Supply Chain Event Model}
The system defines five primary event types in the Ayurvedic herb supply chain:

\begin{enumerate}
\item \textbf{Collection Event:} Initial harvest or wild collection with GPS coordinates
\item \textbf{Processing Event:} Post-harvest treatments including drying, cleaning, and grinding
\item \textbf{Testing Event:} Laboratory quality analysis and certification
\item \textbf{Packaging Event:} Final product packaging with batch information
\item \textbf{Distribution Event:} Transfer between supply chain participants
\end{enumerate}

Each event type has specific data requirements and validation rules enforced through smart contract logic.

\subsection{Herb Batch Lifecycle}

\subsubsection{Collection Phase}
When a collector harvests herbs, the following information is recorded:
\begin{itemize}
\item Herb species identification (scientific and common names)
\item Quantity in kilograms
\item GPS coordinates (latitude and longitude with precision to six decimal places)
\item Geographic metadata (district, state, elevation if available)
\item Collector identification and credentials
\item Harvesting method (wild collection, cultivated, sustainable practices)
\item Environmental conditions (weather, soil type, season)
\item Collection date and time with timezone information
\end{itemize}

A unique batch identifier is generated using the pattern: BATCH-YYYYMMDD-XXXXXXXX, where XXXXXXXX is a cryptographic random string ensuring global uniqueness.

\subsubsection{Blockchain Event Creation}
Upon recording a collection event:
\begin{enumerate}
\item Event data is serialized to JSON format with standardized field names
\item Previous block hash is retrieved (or ``genesis'' for first event)
\item Current timestamp is captured with UTC timezone
\item Combined data is hashed using SHA-256 algorithm
\item Block number is incremented sequentially
\item Event is stored in MongoDB with blockchain metadata
\item Batch record is updated with new event reference
\end{enumerate}

The hash calculation follows the formula:
\begin{equation}
H(n) = \text{SHA256}(\text{event\_data} \parallel H(n-1) \parallel \text{timestamp})
\end{equation}

where $H(n)$ is the hash of the current block, event\_data is the JSON serialization of event information, $H(n-1)$ is the previous block hash, and timestamp is the ISO 8601 formatted event time.

\subsubsection{Processing Phase}
Processing facilities record treatment steps:
\begin{itemize}
\item Processing type (drying, cleaning, grinding, extraction, quality check)
\item Processor identification and facility location
\item Environmental parameters (temperature, humidity, duration)
\item Equipment used and batch size processed
\item Input and output quantities for yield calculation
\item Quality observations and deviation reports
\end{itemize}

Each processing step creates a new blockchain event linked to the batch, maintaining the immutable chain.

\subsubsection{Testing Phase}
Quality assurance laboratories record comprehensive test results:
\begin{itemize}
\item Laboratory name and accreditation details
\item Test types performed (moisture content, heavy metals, pesticides, active compounds, DNA verification)
\item Quantitative results with units and acceptable ranges
\item Pass/fail status for each test parameter
\item Overall grade assignment (Grade A, B, or rejected)
\item Test certificates with unique numbers for external verification
\item Compliance status with regulatory standards (AYUSH, export requirements)
\end{itemize}

Test results are critical for export certification and consumer confidence, hence receive highest priority for data integrity verification.

\subsubsection{QR Code Generation}
Upon completion of processing and testing, a QR code is generated containing:
\begin{itemize}
\item Batch unique identifier
\item Verification hash (truncated SHA-256 of batch data)
\item URL endpoint for provenance access
\item Batch creation date and herb species
\end{itemize}

Consumers scanning this QR code are directed to a web interface displaying the complete provenance chain with blockchain verification status.

\subsection{Smart Contract Logic}
While the proof-of-concept implementation uses server-side validation rather than distributed smart contracts, the business rules enforced include:

\begin{enumerate}
\item \textbf{Sequential Validation:} Processing events cannot be added before collection events
\item \textbf{Quantity Consistency:} Output quantities from processing must not exceed input quantities plus acceptable loss margins
\item \textbf{Temporal Coherence:} Event timestamps must follow chronological ordering
\item \textbf{Role Authorization:} Only authorized entities can record specific event types
\item \textbf{Mandatory Fields:} Critical fields such as location for collection and test results for testing must be present
\item \textbf{Quality Gates:} Batches failing compliance tests are flagged and prevented from proceeding to distribution
\end{enumerate}

\subsection{Blockchain Integrity Verification}
The provenance access interface performs real-time blockchain integrity verification:

\begin{enumerate}
\item Retrieve all events for requested batch ID in chronological order
\item For each event, recalculate hash using stored event data and previous hash
\item Compare calculated hash with stored hash value
\item Verify sequential block numbering without gaps
\item Check that previous hash of event N matches stored hash of event N-1
\item Flag any discrepancies as potential tampering
\end{enumerate}

This verification process ensures end-to-end data integrity and provides cryptographic proof of authenticity to consumers.

\subsection{Stakeholder Roles and Workflows}
\textbf{Collectors} use mobile interface to record harvest events with automatic GPS tagging. Offline capability allows recording in areas with limited connectivity, with subsequent synchronization when network is available.

\textbf{Processors} log each treatment step with environmental parameters. Multiple processing events can be added for complex preparation workflows.

\textbf{Laboratories} enter detailed test results with pass/fail determinations. Integration with laboratory information management systems (LIMS) is supported through API endpoints.

\textbf{Manufacturers} monitor batch progress through dashboard, access complete provenance information, and make decisions based on quality data. Bulk operations for multiple batches are supported.

\textbf{Consumers} scan QR codes on product packaging to view entire supply chain journey, including geographic origins, processing history, and quality certifications. Interface designed for non-technical users with visual timeline presentation.

\textbf{Regulators} query system for compliance audits, verify blockchain integrity, and access aggregated data for sustainability monitoring. Export capabilities in standard formats for regulatory reporting.

% Implementation Details
\section{Implementation Details}

\subsection{Backend Architecture}
The backend is implemented using FastAPI, a modern Python web framework chosen for:
\begin{itemize}
\item High performance with asynchronous request handling
\item Automatic API documentation generation
\item Built-in data validation using Pydantic models
\item Native support for modern Python type hints
\end{itemize}

\subsubsection{Data Models}
Pydantic models define strict schemas for CollectionEvent with geographic coordinates and environmental data, ProcessingEvent with temperature and duration parameters, TestingEvent with nested test results array, HerbBatch with blockchain event references, and BlockchainEvent with hash chaining metadata.

\subsubsection{API Endpoints}
RESTful endpoints provide:
\begin{itemize}
\item POST /api/collection - Create new collection event and batch
\item POST /api/processing - Add processing event to existing batch
\item POST /api/testing - Record laboratory test results
\item GET /api/batch/\{batch\_id\} - Retrieve batch details
\item GET /api/batch/\{batch\_id\}/provenance - Get complete provenance chain with verification
\item GET /api/batches - List all batches with filtering options
\item GET /api/qr/\{batch\_id\} - Generate QR code information
\end{itemize}

\subsubsection{Blockchain Functions}
Core cryptographic functions implemented for hash calculation use JSON serialization with sorted keys to ensure deterministic output regardless of dictionary ordering. SHA-256 produces a 256-bit hash digest encoded as a 64-character hexadecimal string.

\subsubsection{Database Schema}
MongoDB collections include collection\_events for raw collection data with full geographic information, processing\_events for processing step details with parameters, testing\_events for laboratory results with nested test array, blockchain\_events for blockchain metadata with hash chain, and herb\_batches for batch master records with event references. Indices are created on batch\_id and block\_number fields for query performance optimization.

\subsection{Frontend Implementation}
The frontend utilizes React with functional components and hooks for state management.

\subsubsection{Key Pages}
HomePage provides landing page with feature showcase and call-to-action buttons. CollectionPage offers form for recording herb collection with GPS integration. ProcessingPage provides interface for logging processing steps. TestingPage enables laboratory test result entry form. DashboardPage shows overview of all batches with statistics. ScanPage displays consumer-facing provenance display with blockchain verification.

\subsubsection{GPS Integration}
Browser Geolocation API is used for automatic position capture with accuracy typically 5-10 meters in open areas, sufficient for geographic provenance verification.

\subsubsection{API Communication}
Axios library handles HTTP requests with centralized API configuration using environment variables, automatic error handling and retry logic, and request/response interceptors for authentication tokens as future enhancement.

\subsubsection{Responsive Design}
Tailwind CSS utility classes enable mobile-first responsive layouts with breakpoint-based grid systems adapt to screen sizes, touch-friendly button sizes and spacing, and optimized image loading for cellular network conditions.

\subsection{Security Mechanisms}

\subsubsection{Data Validation}
Server-side validation enforces coordinate range checks (latitude: -90 to 90, longitude: -180 to 180), positive quantity values with reasonable upper bounds, date validation preventing future dates for collection events, and enum constraints for herb types and processing types.

\subsubsection{Hash Integrity}
SHA-256 cryptographic hashing provides collision resistance with probability of two different inputs producing same hash being negligible, preimage resistance making it computationally infeasible to reverse hash to original data, and avalanche effect where small change in input produces dramatically different hash.

\subsubsection{Access Control}
Current implementation provides basic authentication framework with API key-based access for different stakeholder roles (planned enhancement), role-based permissions for event creation, and audit logging of all modifications.

\subsection{Offline Synchronization Strategy}
For deployment in areas with intermittent connectivity, browser localStorage API caches pending collection events with timestamp queue, user credentials for offline operation, and previously accessed batch data for reference. Upon connectivity restoration, the system detects online status using navigator.onLine and periodic connectivity checks, queues pending events with conflict detection, uploads in chronological order maintaining temporal consistency, and verifies successful blockchain event creation before clearing queue.

\subsection{Scalability Considerations}
MongoDB indexing strategy includes compound index on (batch\_id, block\_number) for provenance queries, text index on herb\_type for search functionality, and geospatial index on location coordinates for regional queries. Frequently accessed data is cached including recent batch lists with 5-minute TTL, provenance chains for popular consumer queries, and statistics aggregations updated hourly. API rate limiting provides protection against abuse with request throttling of 100 requests per minute per IP address, burst allowance of 20 requests in 10-second window, and graduated backoff for repeated violations.

% Results and Discussion
\section{Results and Discussion}

\subsection{Proof-of-Concept Deployment}
The system was deployed as a proof-of-concept implementation demonstrating end-to-end functionality for Ashwagandha herb tracking. The deployment architecture consisted of backend service running on containerized FastAPI application, MongoDB instance for data persistence, React frontend served through web server, and deployment on cloud infrastructure for accessibility testing.

\subsection{Functional Validation}
The system successfully recorded collection events with automatic GPS coordinate capture with 5-8 meter accuracy, batch ID generation with cryptographic randomness, blockchain event creation with proper hash chaining, and average event recording latency of 450 milliseconds. Multiple processing steps were added to sample batches including drying event with temperature and duration parameters, cleaning event with equipment details, grinding event with output quantity tracking, and quality check event with visual inspection notes. Laboratory testing events recorded moisture content test (target less than 12 percent), heavy metal screening (lead, cadmium, mercury), pesticide residue analysis, DNA barcoding for species verification, and overall grade assignment based on test results. Consumer interface successfully displayed complete event timeline with visual indicators, geographic map showing collection location, test results with pass/fail status, and blockchain verification confirmation.

\subsection{Blockchain Integrity Verification}
Integrity testing performed on sample batches created test batch with 5 sequential events, retrieved provenance chain and verified all hashes, artificially modified stored event data, re-verified chain and successfully detected tampering, and confirmed integrity verification process works correctly. All unmodified chains showed 100 percent verification success. Modified data was detected with 100 percent accuracy, confirming the immutability guarantee of the hash chain.

\subsection{Performance Metrics}
Observed system performance is shown in Table~\ref{tab:performance}.

\begin{table}[htbp]
\caption{Performance Metrics}
\begin{center}
\begin{tabular}{|l|c|}
\hline
\textbf{Metric} & \textbf{Value} \\
\hline
Event Creation Latency & 350-550 ms \\
Provenance Retrieval & 200-400 ms \\
Blockchain Verification & 50-100 ms per event \\
Dashboard Load Time & 800-1200 ms \\
GPS Coordinate Accuracy & 5-10 meters \\
Hash Calculation Time & Less than 5 ms \\
\hline
\end{tabular}
\label{tab:performance}
\end{center}
\end{table}

These performance characteristics demonstrate feasibility for real-world deployment with acceptable user experience.

\subsection{Benefits Demonstrated}
The system provides complete visibility into supply chain where consumers can verify geographic origin of herbs, processing history reveals treatment methods, quality test results provide confidence in product safety, and blockchain verification confirms data authenticity. The system provides multiple fraud deterrents through immutable records that prevent retroactive modification, GPS tagging makes falsified origins detectable, hash chain ensures consistency across events, and quality gates prevent substandard products from proceeding. While the proof-of-concept did not encounter actual fraud attempts, the technical mechanisms are designed to raise the cost and complexity of fraudulent activities significantly. Regulatory compliance features include complete audit trail for export certification, quality test documentation readily accessible, batch recall capability through provenance mapping, and sustainability monitoring through geographic data aggregation. The system facilitates stakeholder coordination through shared visibility that reduces information asymmetry, event-based updates provide real-time status, standardized data formats enable interoperability, and trust establishment through cryptographic verification.

\subsection{User Feedback}
Limited user testing was conducted with student volunteers role-playing different stakeholders. Collectors found GPS capture straightforward but requested offline reliability improvements and battery optimization. Processors appreciated standardized forms but suggested more flexibility for varying processing workflows. Laboratories noted that test result entry could benefit from direct LIMS integration rather than manual entry. Consumers responded positively to provenance visibility but suggested simplified technical language in the interface.

\subsection{Comparative Analysis}
Compared to traditional paper-based systems, advantages include immediate data availability vs. days for document transfer, tamper-evident records vs. easily forged papers, searchable database vs. manual file searching, and consumer access vs. no transparency. Limitations include requirement for digital literacy and device access, dependency on connectivity for real-time updates, initial setup cost vs. minimal paper cost, and ongoing maintenance requirements. Compared to centralized database systems, advantages include cryptographic integrity verification, distributed trust model, transparency to all authorized parties, and reduced single-point-of-failure risk. Limitations include more complex architecture, higher computational overhead, and steeper learning curve for developers.

\subsection{Sustainability Impact Assessment}
Geographic data aggregation enables identification of collection hotspots, monitoring of collection frequency in specific regions, early warning for endangered species pressure, and informed decisions for sustainable sourcing. Historical data reveals optimal harvest timing for maximum yield, seasonal availability patterns, regional variations in herb characteristics, and climate impact on quality parameters. Traceability reduces waste through quality issues detected earlier in chain, targeted recalls minimize disposal, demand forecasting improves planning, and reduced need for intermediary verification.

% Challenges and Limitations
\section{Challenges and Limitations}

\subsection{Technical Challenges}
Many herb collection areas lack reliable cellular or internet connectivity. Offline data capture is implemented but requires subsequent synchronization. GPS coordinates can be captured offline but event recording must wait. Batch synchronization creates potential for errors and duplicates. Real-time provenance updates are not possible in disconnected scenarios. Current implementation faces scaling challenges where MongoDB performance degrades with millions of events without sharding, hash verification becomes computationally expensive for long chains, frontend performance suffers with large batch lists, and storage costs for geo-tagged photos and documents grow linearly. Connecting with existing systems is challenging as laboratory LIMS systems use varied protocols and data formats, legacy ERP systems in manufacturing facilities lack API capabilities, government regulatory portals require manual data entry, and payment and settlement systems operate independently.

\subsection{Adoption Challenges}
Target users have varying technical proficiency. Small-scale collectors may lack smartphone access. Older stakeholders are less comfortable with digital interfaces. Training requirements create initial adoption barrier. User interface must accommodate low-literacy users. Existing supply chain participants may resist transparency. Intermediaries are concerned about disintermediation. Some processors prefer opacity for competitive advantage. Upfront cost concerns exist despite long-term benefits. Change management challenges exist in traditional industry. Comprehensive data capture requires effort as collectors must spend time recording details, processing facilities need to log every step, laboratories face additional data entry work, and balance is needed between data richness and user burden.

\subsection{Data Quality Issues}
Blockchain ensures immutability but not initial accuracy. Collector could falsify herb species identification. GPS coordinates could be captured at wrong location. Quantity measurements could be inaccurate. Quality of input data determines system value. Mitigation strategies are needed including training and certification for data entry personnel, periodic audits with physical verification, reputation systems for data providers, and incentive structures for accurate reporting. Despite validation, errors occur including incorrect herb species selection from dropdown, transposed digits in quantity entry, wrong batch ID selected for subsequent events, and timestamp errors from misconfigured devices.

\subsection{Economic Constraints}
Deployment requires investment in smartphone or tablet devices for collectors, server infrastructure and maintenance, internet connectivity in processing facilities, training programs for all stakeholders, and ongoing technical support. Traceability overhead may not justify for very small quantity collections (less than 1 kg), low-value herbs with minimal adulteration risk, local consumption without export requirements, and collectors operating at subsistence level.

\subsection{Regulatory and Legal Challenges}
Personal information of collectors raises privacy concerns. Privacy regulations may restrict GPS coordinate sharing. Competitive intelligence from processing details raises concerns. Consent requirements exist for blockchain recording. Right to deletion conflicts with immutability. Blockchain records need legal recognition for admissibility in dispute resolution, regulatory acceptance for export certification, standards for blockchain audit trails, and liability for system failures.

\subsection{Technological Limitations}
Current implementation is hash-chained database lacking distributed consensus mechanism, no peer-to-peer network of nodes, centralized control by system operator, and simplified architecture trades off full decentralization. This design decision was made for pragmatic reasons given resource constraints and pilot nature of project, but represents a limitation compared to true permissioned blockchain frameworks. Server-side validation rather than distributed smart contracts means rules are enforced by application code not consensus, updates require server deployment not contract upgrade, less tamper-resistant than distributed contract execution, and dependent on trust in system operator.

\subsection{Deployment Readiness}
The proof-of-concept demonstrates technical feasibility but requires significant enhancements for production deployment including integration with existing systems and databases, enhanced security including encryption and access control, comprehensive testing under real field conditions, scalability improvements for high-volume operations, disaster recovery and backup mechanisms, regulatory compliance verification, and stakeholder onboarding and training programs.

% Future Scope
\section{Future Scope}

\subsection{Technology Enhancements}
Migration to enterprise blockchain platform with Hyperledger Fabric for distributed consensus, peer nodes operated by major stakeholders, distributed smart contract execution, and Byzantine fault tolerance for resilience. Automated data capture through sensors includes temperature and humidity monitoring during drying, soil moisture and pH sensors at collection sites, weighing scales with direct API integration, and automated photo capture at processing steps. Machine learning for quality prediction includes image recognition for species identification from photos, anomaly detection in sensor data streams, quality parameter prediction from environmental conditions, and optimal harvest timing recommendations. Replace centralized file storage with InterPlanetary File System for document storage, content addressing ensures file integrity, distributed storage reduces single-point-of-failure, and lower storage costs for large files.

\subsection{Functional Expansions}
Extend beyond single species pilot to support hundreds of herb species, species-specific quality parameters, customizable processing workflows per herb, and comparative analytics across species. Blockchain-based payment and settlement includes smart contract-triggered payments upon quality verification, transparent pricing visible to all participants, reduced payment delays through automation, and microfinance integration for small collectors. Direct integration with regulatory bodies enables automated export certificate generation, real-time compliance verification, digital signatures from regulatory authorities, and international standard compliance (ISO, GACP). Supply chain participants can transact on platform where buyers can search for specific herbs and qualities, automated matching based on requirements, reputation systems for reliable suppliers, and price discovery through transparent marketplace.

\subsection{Sustainability Enhancements}
Advanced monitoring for endangered species includes geographic heat maps of collection density, automated alerts for over-harvesting, integration with biodiversity databases, and recommendations for alternative sourcing. Environmental impact measurement includes transportation distance calculations, processing energy consumption tracking, carbon offset integration, and sustainability scoring for consumer awareness. Support sustainable practices through cultivation vs. wild collection tracking, organic certification integration, soil health monitoring, and biodiversity impact assessment.

\subsection{User Experience Improvements}
Native apps for better performance include iOS and Android native applications, better offline functionality, push notifications for event updates, and improved GPS and camera integration. Accessibility for diverse users includes interface in Hindi, Tamil, Telugu, Kannada, Marathi, voice-based data entry for low-literacy users, localized herb naming conventions, and cultural adaptation of interface design. Better data presentation includes interactive maps showing herb journey, timeline visualization of events, quality trend charts over time, and supply chain network graphs.

\subsection{Research Extensions}
Game-theoretic analysis includes optimal reward structures for data providers, penalty mechanisms for false information, reputation system economics, and token-based incentives if appropriate. Balancing transparency with confidentiality includes zero-knowledge proofs for selective disclosure, differential privacy for aggregate statistics, encrypted storage with authorized access, and anonymization techniques for sensitive data. Cross-system compatibility includes standard data exchange formats, API specifications for third-party integration, blockchain interoperability protocols, and international collaboration for global supply chains.

\subsection{Pilot Expansion}
Expand to multiple regions including additional districts and states, diverse agro-climatic zones, varied collection practices, and regional language and practice adaptation. Include more participant types such as small-scale cultivators in addition to collectors, Ayurvedic medicine manufacturers, retail pharmacies, export companies, and research institutions. Beyond Ashwagandha pilot include high-value herbs like Saffron and Brahmi, endangered species requiring conservation, commonly used herbs with adulteration problems, and export-oriented species with strict quality requirements.

\subsection{Government Integration}
Formal integration with government systems enables bidirectional data exchange with regulatory databases, compliance verification automation, export permit processing, and subsidy and support scheme linkage. Certification support includes GACP compliance tracking, audit trail for certification bodies, digital certificate issuance, and ongoing compliance monitoring.

% Conclusion
\section{Conclusion}
This paper presented a blockchain-based geo-tagged traceability system specifically designed for Ayurvedic herbal supply chains, addressing critical challenges of transparency, authenticity, and sustainability. The proof-of-concept implementation demonstrated the technical feasibility of combining cryptographic hash chaining, GPS-based location tagging, and smart contract validation to create an immutable record of herb provenance from collection through final product packaging.

The system architecture employs a pragmatic approach suitable for resource-constrained environments, utilizing FastAPI backend, React frontend, and MongoDB storage to provide sub-second event recording latency while maintaining blockchain integrity verification. The implementation successfully tracked sample Ashwagandha batches through collection, processing, and testing phases, with complete provenance chains accessible to consumers through QR code scanning.

Key contributions include the event-based data model supporting diverse supply chain activities, integration of geographic provenance with quality testing, and consumer-facing transparency through simplified interfaces. Performance evaluation indicated average event creation latency of 450 milliseconds and successful blockchain integrity verification, demonstrating practical viability.

However, significant challenges remain for production deployment, including connectivity constraints in rural collection areas, stakeholder adoption barriers, data entry point vulnerabilities, and scalability limitations. The current implementation represents a simplified blockchain architecture rather than a fully distributed permissioned ledger, a pragmatic trade-off for proof-of-concept development.

Future work should focus on migration to enterprise blockchain platforms such as Hyperledger Fabric, integration with IoT sensors for automated data capture, and expansion to multiple herb species and geographic regions. Additionally, the incorporation of artificial intelligence for quality prediction and formal integration with government regulatory portals would significantly enhance system utility.

The demonstrated system provides a foundation for transforming transparency and trust in Ayurvedic herbal supply chains, supporting sustainable harvesting practices, enabling regulatory compliance, and ultimately protecting consumer health through verifiable product authenticity. While deployment challenges are non-trivial, the technical feasibility established through this work suggests that blockchain-based traceability represents a viable path forward for modernizing traditional medicine supply chains in India and globally.

% References
\begin{thebibliography}{99}

\bibitem{nakamoto2008bitcoin}
S. Nakamoto, ``Bitcoin: A Peer-to-Peer Electronic Cash System,'' 2008.

\bibitem{kshetri2018blockchain}
N. Kshetri, ``Blockchain's roles in meeting key supply chain management objectives,'' \textit{International Journal of Information Management}, vol. 39, pp. 80-89, 2018.

\bibitem{tian2016agri}
F. Tian, ``An agri-food supply chain traceability system for China based on RFID blockchain technology,'' in \textit{2016 13th International Conference on Service Systems and Service Management (ICSSSM)}, pp. 1-6, 2016.

\bibitem{christidis2016blockchain}
K. Christidis and M. Devetsikiotis, ``Blockchains and Smart Contracts for the Internet of Things,'' \textit{IEEE Access}, vol. 4, pp. 2292-2303, 2016.

\bibitem{azaria2016medrec}
A. Azaria, A. Ekblaw, T. Vieira, and A. Lippman, ``MedRec: Using Blockchain for Medical Data Access and Permission Management,'' in \textit{2016 2nd International Conference on Open and Big Data (OBD)}, pp. 25-30, 2016.

\bibitem{pournader2020blockchain}
M. Pournader, Y. Shi, S. Seuring, and S. L. Koh, ``Blockchain applications in supply chains, transport and logistics: a systematic review of the literature,'' \textit{International Journal of Production Research}, vol. 58, no. 7, pp. 2063-2081, 2020.

\bibitem{hyperledger2020}
Hyperledger Fabric Documentation, ``Blockchain for Business - An Introduction to Hyperledger Technologies,'' The Linux Foundation, 2020.

\bibitem{malik2019trustchain}
S. Malik, V. Dedeoglu, S. S. Kanhere, and R. Jurdak, ``TrustChain: Trust Management in Blockchain and IoT supported Supply Chains,'' in \textit{2019 IEEE International Conference on Blockchain (Blockchain)}, pp. 184-193, 2019.

\bibitem{casino2019systematic}
F. Casino, T. K. Dasaklis, and C. Patsakis, ``A systematic literature review of blockchain-based applications: Current status, classification and open issues,'' \textit{Telematics and Informatics}, vol. 36, pp. 55-81, 2019.

\bibitem{cruz2018rbac}
J. P. Cruz, Y. Kaji, and N. Yanai, ``RBAC-SC: Role-Based Access Control Using Smart Contract,'' \textit{IEEE Access}, vol. 6, pp. 12240-12251, 2018.

\bibitem{pilkington2016blockchain}
M. Pilkington, ``Blockchain Technology: Principles and Applications,'' in \textit{Research Handbook on Digital Transformations}, edited by F. Xavier Olleros and Majlinda Zhegu, Edward Elgar, 2016.

\bibitem{gervais2016security}
A. Gervais, G. O. Karame, K. Wüst, V. Glykantzis, H. Ritzdorf, and S. Capkun, ``On the Security and Performance of Proof of Work Blockchains,'' in \textit{Proceedings of the 2016 ACM SIGSAC Conference on Computer and Communications Security}, pp. 3-16, 2016.

\bibitem{gatteschi2018blockchain}
V. Gatteschi, F. Lamberti, C. Demartini, C. Pranteda, and V. Santamaría, ``Blockchain and Smart Contracts for Insurance: Is the Technology Mature Enough?,'' \textit{Future Internet}, vol. 10, no. 2, p. 20, 2018.

\bibitem{ayush2015gacp}
Ministry of AYUSH, Government of India, ``Good Agricultural and Collection Practices (GACP) for Medicinal Plants,'' 2015.

\bibitem{wood2014ethereum}
G. Wood, ``Ethereum: A secure decentralised generalised transaction ledger,'' \textit{Ethereum project yellow paper}, vol. 151, pp. 1-32, 2014.

\bibitem{zheng2018blockchain}
Z. Zheng, S. Xie, H. N. Dai, X. Chen, and H. Wang, ``Blockchain challenges and opportunities: A survey,'' \textit{International Journal of Web and Grid Services}, vol. 14, no. 4, pp. 352-375, 2018.

\bibitem{cole2019blockchain}
R. Cole, M. Stevenson, and J. Aitken, ``Blockchain technology: implications for operations and supply chain management,'' \textit{Supply Chain Management: An International Journal}, vol. 24, no. 4, pp. 469-483, 2019.

\bibitem{salah2019blockchain}
K. Salah, M. H. Rehman, N. Nizamuddin, and A. Al-Fuqaha, ``Blockchain for AI: Review and Open Research Challenges,'' \textit{IEEE Access}, vol. 7, pp. 10127-10149, 2019.

\bibitem{rejeb2019leveraging}
A. Rejeb, J. G. Keogh, and H. Treiblmaier, ``Leveraging the Internet of Things and Blockchain Technology in Supply Chain Management,'' \textit{Future Internet}, vol. 11, no. 7, p. 161, 2019.

\bibitem{queiroz2019blockchain}
M. M. Queiroz and S. Fosso Wamba, ``Blockchain adoption challenges in supply chain: An empirical investigation of the main drivers in India and the USA,'' \textit{International Journal of Information Management}, vol. 46, pp. 70-82, 2019.

\bibitem{saberi2019blockchain}
S. Saberi, M. Kouhizadeh, J. Sarkis, and L. Shen, ``Blockchain technology and its relationships to sustainable supply chain management,'' \textit{International Journal of Production Research}, vol. 57, no. 7, pp. 2117-2135, 2019.

\bibitem{sharma2018software}
P. K. Sharma, M. Y. Chen, and J. H. Park, ``A Software Defined Fog Node Based Distributed Blockchain Cloud Architecture for IoT,'' \textit{IEEE Access}, vol. 6, pp. 115-124, 2018.

\bibitem{yuan2016blockchain}
Y. Yuan and F. Y. Wang, ``Towards blockchain-based intelligent transportation systems,'' in \textit{2016 IEEE 19th International Conference on Intelligent Transportation Systems (ITSC)}, pp. 2663-2668, 2016.

\bibitem{mettler2016blockchain}
M. Mettler, ``Blockchain technology in healthcare: The revolution starts here,'' in \textit{2016 IEEE 18th International Conference on e-Health Networking, Applications and Services (Healthcom)}, pp. 1-3, 2016.

\bibitem{underwood2016blockchain}
S. Underwood, ``Blockchain beyond bitcoin,'' \textit{Communications of the ACM}, vol. 59, no. 11, pp. 15-17, 2016.

\end{thebibliography}

\end{document}
